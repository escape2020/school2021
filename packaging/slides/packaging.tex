\PassOptionsToPackage{
  unicode,
  pdfusetitle,
  colorlinks,
  linkcolor=vertexDarkRed,
  urlcolor=vertexDarkRed,
  citecolor=vertexDarkRed,
}{hyperref}
\documentclass[
  aspectratio=1610,
]{beamer}
\usetheme{vertex}
\usepackage[american]{babel}

\usepackage[autostyle]{csquotes}

\usepackage{fontspec}
\usepackage{graphicx}

\usepackage[outputdir=build]{minted}
\usepackage{xcolor}
\usepackage{xparse}
\usepackage{tcolorbox}
\tcbuselibrary{minted}
\tcbuselibrary{skins}
\tcbuselibrary{xparse}
\usepackage{accsupp}

\usepackage{bookmark}
\usepackage[shortcuts]{extdash}
\usepackage{fontawesome5}


\definecolor{positive}{HTML}{15B01A}
\colorlet{negative}{vertexDarkRed}
\setmonofont{Fira Mono}

\newcommand\headline[1]{\textbf{\Large\strut{}#1}\newline}
\newcommand\headlineframe[1]{%
  \begin{frame}[c]%
    \begin{center}%
      \Huge\color{vertexDarkRed}#1%
    \end{center}%
  \end{frame}%
}%

\renewcommand\theFancyVerbLine{%
  \BeginAccSupp{method=escape,ActualText={}}%
  {\ttfamily\tiny\arabic{FancyVerbLine}}%
  \EndAccSupp{}%
}%

\tcbset{codebox/.style={
  listing engine=minted,
  colback=black!5!white,
  colframe=vertexDarkGrey,
  listing only,
  left=5mm,
  top=2pt,
  bottom=0pt,
  boxsep=3pt,
  enhanced,
  overlay={%
    \begin{tcbclipinterior}
      \fill[black!20!white] (frame.south west) rectangle ([xshift=5mm]frame.north west);
    \end{tcbclipinterior}
  },
}}

\DeclareTCBInputListing{inputcode} {O{} m m O{}} {
  codebox,
  listing file=#2,
  minted language=#3,
  minted options={
    fontsize=\footnotesize,
    breaklines,
    autogobble,
    linenos,
    numbersep=3mm,
    #4
  },
  #1
}
\DeclareTCBListing{code} {O{} m} {
  codebox,
  minted language=#2,
  minted options={
    fontsize=\footnotesize,
    breaklines,
    autogobble,
    linenos,
    numbersep=3mm
  },
  #1
}

\author[M. Nöthe]{Maximilian Nöthe}
\title[Packaging]{Packaging and Distributing Python Projects}
\date[2021-06-10]{%
  \raisebox{-1.5cm}{\includegraphics[height=3cm]{../../logo-Escape_cropped.png}}
  \Large Summer School – 2021-06-10
}
\institute[TU Dortmund]{Astroparticle Physics, TU Dortmund}

\begin{document}

\maketitle

\begin{frame}[c]{overview}
  \tableofcontents
\end{frame}

\copywarning{}

\section{Introduction}

\begin{frame}[c]{The Python Package Index}
  \begin{itemize}
    \item Python packages are published on the Python Package Index (\url{https://pypi.org})
    \item \mintinline{text}+pip install foo+ will by default:
      \begin{enumerate}
        \item Search for a package named \texttt{foo} on PyPI
        \item Download the best available distribution for your platform
        \item Install all dependencies of the package
        \item Install the package
      \end{enumerate}
    \item There is \url{https://test.pypi.org} for people to test their packaging code before
      publishing to \enquote{the real thing}.
    \item It is also possible to self-host a python package index
  \end{itemize}
\end{frame}

\begin{frame}[c]{Source Distributions and Wheels}

  \begin{columns}[onlytextwidth, t]%
    \begin{column}{0.495\textwidth}%
      \textbf{\Large Source Distributions}
      \begin{itemize}
        \item \texttt{.zip} or \texttt{.tar.gz} archives of the project
        \item Simplest solution to publish your package
        \item If a package contains compiled components, these need to be built at installation time
      \end{itemize}
    \end{column}%
    \hfill%
    \begin{column}{0.495\textwidth}%
      \textbf{\Large Wheels} \includegraphics[height=1cm]{images/cheese.png}
      \begin{itemize}
        \item Standardized format for pre-built python packages
        \item Simple for pure-python packages (no compiled components)
        \item Platform-dependent wheels for packages with compiled components
          \begin{itemize}
            \item C-Extensions
            \item Cython-Code
            \item Wrappers for C or C++-Libraries
            \item \dots
          \end{itemize}
      \end{itemize}
    \end{column}%
  \end{columns}%
\end{frame}

\begin{frame}[c]{Wheels}
  \begin{itemize}
    \item Platform dependent binary wheels must follow standards to be uploaded to PyPI
    \item This is to ensure they run on many systems (not just on your computer)
    \item Essentially:
      \begin{itemize}
        \item Compile using the oldest C-Standard Library a package wants to support
        \item Include all needed libraries in the wheel
      \end{itemize}
  \end{itemize}

  \bigskip
  \begin{center}
    \Large More on how to actually build wheels for your own projects later.
  \end{center}
\end{frame}



\section{Using setuptools}
\headlineframe{Using setuptools}

\begin{frame}[c]{setuptools}
  \begin{itemize}
    \item \texttt{setuptools} is the most common solution for python packaging
    \item Allows to declare package metadata, dependencies
    \item Facilitates creation of files for distribution
  \end{itemize}
\end{frame}
\begin{frame}[c, fragile]{Example Package Structure}
  \begin{columns}[onlytextwidth, c]%
    \begin{column}{0.44\textwidth}%
      \begin{code}{text}
        eschool21_demo
        ├── eschool21_demo
        │   ├── tests
        │   │   ├── __init__.py
        │   │   └── test_fibonacci.py
        │   ├── fibonacci.py
        │   └── __init__.py
        ├── LICENSE
        ├── pyproject.toml
        ├── README.md
        ├── setup.cfg
        └── setup.py
      \end{code}
    \end{column}%
    \hfill%
    \begin{column}{0.55\textwidth}%
      \only<1>{%
        Common convention: project directory equal or very similar to package name:
        \begin{itemize}
          \item \texttt{numpy / numpy}
          \item \texttt{PyTables / tables}
          \item \texttt{python-dateutil / dateutil}
        \end{itemize}
      }%
      \only<2>{%
        Files in the base directory for metadata / build configuration
        \small
        \begin{description}[\texttt{pyproject.toml}]
          \item[\texttt{README.md}] Project description
          \item[\texttt{LICENSE}] Software license
          \item[\texttt{pyproject.toml}] Common configuration for python projects
          \item[\texttt{setup.\{py,cfg\}}] setuptools specific project files
        \end{description}
      }%
    \end{column}%
  \end{columns}%
\end{frame}

\begin{frame}[c, fragile]{pyproject.toml}
  \begin{itemize}
    \item Defines \emph{build-time} dependencies of a python package
    \item Uses the toml file format: \url{https://github.com/toml-lang/toml}
    \item Defined in \href{https://www.python.org/dev/peps/pep-0517/}{PEP 517} and \href{https://www.python.org/dev/peps/pep-0518/}{PEP 518}
    \item Many other tools can also be configured through \texttt{pyproject.toml},\\ e.g. black, poetry, \ldots.
  \end{itemize}

  \inputcode[title={Minimal pyproject.toml file for projects using setuptools}]{../eschool21_demo/pyproject.toml}{toml}
\end{frame}

\begin{frame}[c, fragile]{setup.py and setup.cfg}
  \begin{itemize}
    \item All metadata concerning your package can be specified in a \texttt{setup.py} file
      \begin{itemize}
        \item Using code for configuration is generally not a good idea
        \item Projects can run arbitrary python code in \texttt{setup.py} to setup the project
      \end{itemize}
    \item[$\Rightarrow$] For simple projects, only use the \texttt{setup.cfg}
    \item Editable installs currently require a minimal \texttt{setup.py}:
  \end{itemize}
  \inputcode{../eschool21_demo/setup.py}{python}
\end{frame}

\begin{frame}[c, fragile]{setup.cfg}
  \begin{code}{ini}
    [metadata]
    name = mypackage
    version = 0.1.0
    description = Example Package
    license = MIT
    # ... many more metadata options possible, see docs
    long_description = file: README.md
    long_description_content_type = text/markdown

    classifiers =
      # see https://pypi.org/classifiers/ for more
      License :: OSI Approved :: MIT License

    [options]
    packages = find:  # automatically find python packages
    python_requires = >=3.6
    install_requires =
      astropy >= 4
  \end{code}
\end{frame}

\begin{frame}[c, fragile]{Building the Project}
  \begin{itemize}
    \item Install the \texttt{build} package (already available in the \texttt{eschool21} environment):
      \begin{code}{text}
        $ python -m pip install build
      \end{code}
    \item Run the build module in the project directory
      \begin{code}{text}
        $ python -m build
      \end{code}
    \item You will get both the sdist and the wheel in the \texttt{dist} directory:
      \begin{code}{text}
        $ ls -1 dist
        eschool21_demo-0.1.0-py3-none-any.whl
        eschool21_demo-0.1.0.tar.gz
      \end{code}
  \end{itemize}
\end{frame}


\begin{frame}[fragile, c]{Upload to (Test-)PyPI}
  \begin{itemize}
    \item Create an Account at (Test-)PyPI
    \item Install \texttt{twine} (already available in the \texttt{eschool21} environment)
      \begin{code}{text}
        $ python -m pip install twine
      \end{code}
    \item Run the upload (here to test.pypi.org):
      \begin{code}{text}
        $ twine upload --repository testpypi dist/*
      \end{code}
    \item Go to your uploaded project and check everything is ok

    \onslide<2->{
      \begin{center}
        \Large\color{vertexDarkRed} For security reasons, PyPI does not allow replacing uploaded files.
        You have to upload a new \emph{version}.
      \end{center}
    }
  \end{itemize}
\end{frame}

\begin{frame}[c,fragile]{Entry Points}
  \begin{itemize}
    \item Console script entry points define scripts that get installed so they can be run from the command line

    \begin{code}[title={setup.cfg}]{ini}
    [options.entry_points]
    console_scripts = 
      fibonacci = eschool21_demo.__main__:main
    \end{code}
  \end{itemize}
\end{frame}


\begin{frame}[c,fragile]{Including Data}
  \begin{itemize}
    \item To include non-code files into the source distribution and wheels you need to add
      \begin{code}[title={setup.cfg}]{ini}
      [options]
      include_package_data = True
      \end{code}
    \item And define these additional files in an additional file \texttt{MANIFEST.in}
  \end{itemize}
\end{frame}


\section{Versions and Semantic Versioning}
\headlineframe{Versions and Semantic Versioning}
\begin{frame}[fragile, c]{Versioning your Projects}
  \begin{itemize}
    \item \href{https://www.python.org/dev/peps/pep-0440}{PEP 440} prescribes a versioning scheme for all python projects:
      \begin{code}{text}
      [N!]N(.N)*[{a|b|rc}N][.postN][.devN]
      \end{code}
      \begin{description}[aN|bN|rcN]
      \item[N!] Version epoch, extremely rare, needed only when switching the versioning scheme
      \item[N(.N)*] Version identifier as arbitrarily many numbers separated by a dot
      \item[aN|bN|rcN] Pre-releases (alpha, beta, release candidate) for testing
      \item[.postN] Post releases, no changes to actual code, but e.\,g.\ better docs / fixed build system
      \item[.devN] are development releases (N can be used e.g. to specify the number of commits since the last released)
    \end{description}
    \item By default, pip will not consider pre- and dev-releases
    \item Versions are sortable
  \end{itemize}
\end{frame}

\begin{frame}[fragile, c]{Version examples}
  \begin{code}[title=Versions in sorted order]{text}
    1.0.9
    1.1.0.dev10
    1.1.0a1
    1.1.0a2
    1.1.0b1
    1.1.0rc1
    1.1.0
    1.1.0.post1
    1.2.0
  \end{code}
\end{frame}

\begin{frame}[c]{Semantic Versioning}
  \begin{itemize}
    \item See \url{https://semver.org}

    \item SemVer uses a three part version like this:
      \begin{center}
        \Large\texttt{MAJOR.MINOR.PATCH}
      \end{center}
    \item Projects must increment:
      \begin{enumerate}
        \item \texttt{MAJOR} version when you make incompatible API changes,
        \item \texttt{MINOR} version when you add functionality in a backwards compatible manner
        \item \texttt{PATCH} version when you make backwards compatible bug fixes
      \end{enumerate}
    \item This makes depending on specific versions much easier
  \end{itemize}

  \bigskip
  \begin{center}
    Caveats:\\
    Many python projects do not strictly follow SemVer (e.\,g.\ numpy) \\
    Many projects make breaking changes in MINOR updates until reaching 1.0.0
  \end{center}
\end{frame}

\begin{frame}[fragile, c]{Specifying Versions of Dependencies}
  \begin{itemize}
    \item One of the most important things for packages is defining the compatible versions
  of the depedencies.
    \item Projects can require smaller, larger, exactly equal and \enquote{compatible} versions
    \item Projects can exclude versions
    \item Versions may contain wildcards
    \item Also defined in \href{https://www.python.org/dev/peps/pep-0440/#version-specifiers}{PEP 440}
  \end{itemize}

  \begin{code}[title={Depedency defintions}]{text}
     pandas                 # no requirement on the version
     pandas >=1.0           # at least 1.0
     pandas >=1.0,<2.0.0a0  # at least 1.0 but smaller than 2.0.0a0
     pandas ==1.*           # Any 1.x version
     pandas ~=1.1           # Any 1.x version >=1.1
     pandas ~=1.1.2         # Any 1.1.x version >=1.1.2
     pandas >=1.1,!=1.1.1   # Exclude 1.1.1 (had a bug?)
  \end{code}
\end{frame}

\begin{frame}[c, fragile]{Extras}
  \begin{itemize}
    \item You can use \texttt{extras} to define sets of optional dependencies
    \item Usefull especially for test and docs dependencies
    \item Consider providing an \texttt{all} extra to make it simple
    \item Extras can be requested in \mintinline{python}+[]+
      \begin{code}{bash}
        $ pip install "package[extra1,extra2]"
      \end{code}
  \end{itemize}

  \inputcode[title={\texttt{setup.cfg}}]{./../eschool21_demo/setup.cfg}{ini}[firstline=22, lastline=30]
\end{frame}

\begin{frame}[fragile, c]{Avoiding duplicated version definitions}
  \begin{itemize}
    \item A common problem is that version information is needed at multiple locations
      \begin{itemize}
        \item The git tag
        \item The package version (e.\,g.\ in \texttt{setup.py} or \texttt{CMakeLists.txt})
        \item Accessible version in the code (e.\,g.\ \mintinline{python}{eschool21_demo.__version__})
      \end{itemize}
    \item Not having this information duplicated avoids errors
    \item Setuptools supports reading this from the code
    \item Tools like \texttt{setuptools\_scm} can extract version information from git tags, \\
      but is a bit complicated to setup correctly
  \end{itemize}
\end{frame}

\begin{frame}[fragile, c]{Defining the version in code}
  \inputcode[title=\mintinline{text}+eschool21_demo/__init__.py+]{../eschool21_demo/eschool21_demo/__init__.py}{python}

  \begin{code}[title=setup.cfg]{ini}
    [metadata]
    version = attr: eschool21_demo.__version__
  \end{code}

  This also works when \mintinline{text}+__init__.py+ already imports dependencies,
  since setuptools is not actually importing the variable but parses the code.
\end{frame}


\section{Choosing a License}
\headlineframe{Choosing a License}

\begin{frame}[c]{Software Licenses}
  \begin{itemize}
    \item Disclaimer: I am a Physicist, not a Lawyer
    \item Software licenses have two main purposes
      \begin{enumerate}
        \item Define what other people are allowed to do with your software
        \item Free the authors from liability / waving warranties
      \end{enumerate}
    \item There are several \enquote{standard} free and open source licenses,\\
      endorsed by the \emph{Open Source Initiative}: \url{https://opensource.org/licenses}
    \item These licenses range from
      \begin{itemize}
        \item very short to very long
        \item very restrictive to very permissive
      \end{itemize}
    \item \enquote{Free as in freedom, not as in free beer.}
  \end{itemize}
\end{frame}

\begin{frame}[c]{The MIT License}
  \small
  \input{../../LICENSE}
\end{frame}

\begin{frame}[c]{GPL and LGPL}
  \begin{itemize}
    \item Allows redistribution, modification, running, ...
    \item Requires that source code is always published for binary distributions
    \item Requires that derivative works are licensed under the same or a compatible license (copyleft)
    \item If linking linking libraries constitutes a derivative work is contentious
    \item The LGPL explicitly allows proprietary software to link LGPL licensed libraries
  \end{itemize}
\end{frame}

\begin{frame}[c]{Scientific Software}
  \begin{itemize}
    \item Opinion: The scientific method requires all code and data to be accessible
      \begin{itemize}
        \item Reproducibility
        \item Peer review
      \end{itemize}
    \item This is most often not the case, but starting to get traction
      \begin{itemize}
        \item Journals requiring release of software and data alongside publications
        \item General trend towards more open development / open source scientific software
        \item \enquote{Replication Crisis}
      \end{itemize}
  \end{itemize}
\end{frame}

\section{Publishing Binary Wheels}
\headlineframe{Publishing Binary Wheels}

\begin{frame}[c]{Example Project using cython and numpy}
  \begin{itemize}
    \item Demo project in \mintinline{text}+school21/packaging/eschool21_cython_demo+
    \item Using cython and numpy's cython API to build a C-extension
    \item We need Cython and numpy at build time
    \item We need to compile against the oldest supported numpy version
      \inputcode[title=\texttt{pyproject.toml}]{../eschool21_cython_demo/pyproject.toml}{toml}
  \end{itemize}
\end{frame}

\begin{frame}[c]{Publishing Binary Wheels}
  \headline{\faIcon{windows} Windows}
  With python >= 3.5, things got a lot easier
  \begin{enumerate}
    \item Install Visual Studio C/C++ Compilers and Windows SDK
    \item Install all versions of python you want to support
    \item Build the wheel for each version of python
    \item Upload to PyPI
  \end{enumerate}
\end{frame}


\begin{frame}[c]{Publishing Binary Wheels}
  \headline{\faIcon{apple} macOS}
  Decide the oldest supported macOS version (10.9 is most common)

  \begin{enumerate}
    \item Install compilers using \texttt{xcode install --select}
    \item Install all versions of python you want to support (e.g. using pyenv or brew)
    \item \mintinline{bash}+export MACOSX_DEPLOYMENT_TARGET=10.9+
    \item Build the wheel for each version of python
    \item Upload to PyPI
  \end{enumerate}
\end{frame}

\begin{frame}[c]{Publishing Binary Wheels}
  \headline{\faIcon{linux} Linux}
  \begin{itemize}
    \item The many different Linux distributions and versions make things a bit harder
    \item Standardized wheels to make sure they run on \enquote{manylinux} distributions
    \item Essentially you have to compile with the oldest glibc you want to support
    \item e.\,g.\ \texttt{manylinux2014} is based on CentOS 7, glibc 2.17
  \end{itemize}

  The python packaging authority (PyPA) provides docker containers for each of these standards,
  which is the best way of building these, see \url{https://github.com/pypa/python-manylinux-demo}.
\end{frame}

\begin{frame}[c, fragile]{Publishing Binary Wheels – Including dependencies}
  \begin{itemize}
    \item Binary wheels are only allowed to link externally against basic system libraries defined in \href{https://www.python.org/dev/peps/pep-0513/}{PEP 513}/\href{https://www.python.org/dev/peps/pep-0599/}{PEP 599}.
    \item All other libraries must be included in the wheel
    \item \href{https://github.com/pypa/auditwheel}{\texttt{auditwheel}} (\faIcon{linux}) and \href{https://github.com/matthew-brett/delocate}{\texttt{delocate}} (\faIcon{apple}) take care of this
    \item No off-the-shelve solution for \faIcon{windows}
    \item \href{https://github.com/pypa/cibuildwheel}{\texttt{cibuildwheel}} offers CI build configurations for wheels for all platforms
  \end{itemize}

  \begin{code}[title={Inside the manylinux docker container}]{bash}
    $ auditwheel repair --plat manylinux2014_x86_64 <wheel> -w <outputdir>
  \end{code}
\end{frame}


\section{A new Alternative: poetry}
\headlineframe{A new Alternative: poetry}
\begin{frame}[c]{Poetry}
  poetry aims to provide a complete solution for dependency management and packaging

  \begin{itemize}
    \item[\color{positive}\faPlus] Configured completely in \texttt{pyproject.toml}
    \item[\color{positive}\faPlus] Automatic creation of virtual environments with all dependencies
    \item[\color{positive}\faPlus] Exact versions of all dependencies (including transitive) using a \enquote{lock file}
    \item[\color{positive}\faPlus] Building and publishing packages
    \item[\color{positive}\faPlus] Initial setup of new projects
    \item[\color{negative}\faMinus] No support for binary extensions / wheels yet
  \end{itemize}
\end{frame}

\headlineframe{live demo}

\section{Conda Packages and conda-forge}
\headlineframe{Conda Packages and conda-forge}

\begin{frame}[c, fragile]{Conda Packages}
  \begin{itemize}
    \item Conda packages for python packages should always start from a buildable package using the tools introduced before
    \item Then, we only need to define build- and runtime depedencies as well as metadata in a yaml file \texttt{meta.yaml}
      \begin{code}{bash}
        $ conda build <path to recipe directory>
        $ anaconda upload <path to package>
      \end{code}
    \item \href{https://conda-forge.org/}{\texttt{conda-forge}} provides CI infrastructure to automatically build conda packages for open source projects
  \end{itemize}
\end{frame}

\section{Conclusions and Recommendations}
\headlineframe{Conclusions and Recommendations}

\begin{frame}[c]{Conclusions and Recommendations}
  \begin{itemize}
    \item Always make sure your code is a valid python package and declares its dependencies
    \item Publishing sdists / any-wheels to PyPI is free and easy \\
      → your code is just a \texttt{pip install} away
    \item Choose a permissive FOSS License for scientific software (MIT or BSD 3-Clause)
    \item When you expect your users to rely on conda, also publish conda packages
    \item conda-forge is highly recommended, as it automatizes this process greatly
    \item When using compiled python extensions, consider publishing wheels and/or conda packages \\
      → Much easier installation for your users, but may be a bit complicated to setup
    \item Consider using poetry instead of setuptools for applications / libraries without compiled components
    \item The python packaging landscape has improved greatly in the last couple of years, check carefully if guides
      you find are still up-to-date
  \end{itemize}
\end{frame}

\begin{frame}[c]{Further Reading}
  \begin{description}[conda build docs]
    \item[PyPA User Guide] \url{https://packaging.python.org}
    \item[setuptools docs] \url{https://setuptools.readthedocs.io/}
    \item[poetry] \url{https://python-poetry.org/}

    \medskip
    \item[conda-forge] \url{https://conda-forge.org/}
    \item[conda build docs] \url{https://docs.conda.io/projects/conda-build}

    \medskip
    \item[cibuildwheel] \url{https://github.com/pypa/cibuildwheel}
    \item[auditwheel] \url{https://github.com/pypa/auditwheel}
    \item[delocate] \url{https://github.com/matthew-brett/delocate}
    \item[manylinux demo] \url{https://github.com/pypa/python-manylinux-demo}
  \end{description}
\end{frame}

\end{document}
