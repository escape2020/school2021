\PassOptionsToPackage{
  unicode,
  pdfusetitle,
  colorlinks,
  linkcolor=vertexDarkRed,
  urlcolor=vertexDarkRed,
  citecolor=vertexDarkRed,
}{hyperref}
\documentclass[
  aspectratio=1610,
]{beamer}
\usetheme{vertex}
\usepackage[american]{babel}

\usepackage[autostyle]{csquotes}

\usepackage{fontspec}
\usepackage{graphicx}

\usepackage[outputdir=build]{minted}
\usepackage{xcolor}
\usepackage{xparse}
\usepackage{tcolorbox}
\tcbuselibrary{minted}
\tcbuselibrary{skins}
\tcbuselibrary{xparse}
\usepackage{accsupp}

\usepackage{bookmark}
\usepackage[shortcuts]{extdash}
\usepackage{fontawesome5}


\definecolor{positive}{HTML}{15B01A}
\colorlet{negative}{vertexDarkRed}
\setmonofont{Fira Mono}

\newcommand\headline[1]{\textbf{\Large\strut{}#1}\newline}
\newcommand\headlineframe[1]{%
  \begin{frame}[c]%
    \begin{center}%
      \Huge\color{vertexDarkRed}#1%
    \end{center}%
  \end{frame}%
}%

\renewcommand\theFancyVerbLine{%
  \BeginAccSupp{method=escape,ActualText={}}%
  {\ttfamily\tiny\arabic{FancyVerbLine}}%
  \EndAccSupp{}%
}%

\tcbset{codebox/.style={
  listing engine=minted,
  colback=black!5!white,
  colframe=vertexDarkGrey,
  listing only,
  left=5mm,
  top=2pt,
  bottom=0pt,
  boxsep=3pt,
  enhanced,
  overlay={%
    \begin{tcbclipinterior}
      \fill[black!20!white] (frame.south west) rectangle ([xshift=5mm]frame.north west);
    \end{tcbclipinterior}
  },
}}

\DeclareTCBInputListing{inputcode} {O{} m m O{}} {
  codebox,
  listing file=#2,
  minted language=#3,
  minted options={
    fontsize=\footnotesize,
    breaklines,
    autogobble,
    linenos,
    numbersep=3mm,
    #4
  },
  #1
}
\DeclareTCBListing{code} {O{} m} {
  codebox,
  minted language=#2,
  minted options={
    fontsize=\footnotesize,
    breaklines,
    autogobble,
    linenos,
    numbersep=3mm
  },
  #1
}

\author[M. Nöthe]{Maximilian Nöthe}
\title[Testing]{Unit Testing \\ Test Driven Development \\ Continuous Integration}
\date[2021-06-10]{%
  \raisebox{-1.5cm}{\includegraphics[height=3cm]{../../logo-Escape_cropped.png}}
  \Large Summer School – 2021-06-10
}
\institute[TU Dortmund]{Astroparticle Physics, TU Dortmund}

\begin{document}
\maketitle

\begin{frame}[c]{Overview}
  \tableofcontents
\end{frame}

\copywarning{}

\section{Introduction}
\headlineframe{Introduction}

\begin{frame}[c]{Automated Software Testing}
  \begin{itemize}
    \item Verifying that a software works as intended is crucial
    \item Doing this manually using whatever method you can think of
      \begin{itemize}
        \item is very tedious
        \item is errorprone
        \item will result in the tests not being done most of the time
      \end{itemize}
    \item[$\Rightarrow$] We need automated tests that verify our software
    \item Tests fall into three categories
      \begin{enumerate}
        \item Unit tests
        \item Integration tests
        \item Performance tests
      \end{enumerate}
  \end{itemize}
\end{frame}

\begin{frame}[c]{Unit tests}
  \begin{itemize}
    \item Test single \enquote{units} of the code in isolation
    \item Require modular design of the code base
    \item Are the bedrock of any more complicated tests
    \item Must be fast and easy to run $⇒$ or they would not be run most of the time
  \end{itemize}
\end{frame}

\begin{frame}[c]{Properties of good unit tests}
  \begin{description}[Demonstrability]
    \item[Existence] \faSmileWink[regular]
    \item[Correctness] The code under test behaves according to requirements / specifications
    \item[Completeness] The tests cover all required features / use cases
    \item[Readability] Writing tests for tests would result in infinite recursion\\
      $⇒$ tests must readable, so they can be easily verified by inspection
    \item[Demonstrability] Good tests show how your code is meant to be used
    \item[Resilience] Tests should only fail if what they test breaks
  \end{description}
\end{frame}

\begin{frame}[c]{Frameworks}
  All modern languages have one or more frameworks for tests, a small selection:

  \begin{description}[Python]
    \item[Python] pytest
    \item[C++] Catch2, GoogleTest
    \item[Java] JUnit
    \item[Rust] Part of the language
    \item[Julia] \texttt{Test} module in the standard library
  \end{description}
\end{frame}

\begin{frame}[c]{Integration tests}
  \begin{itemize}
    \item Test that multiple \emph{units} are working together
    \item E.\,g.\ testing a whole command line application
    \item Can grow arbitrarily large / complicated
  \end{itemize}
\end{frame}

\begin{frame}[c]{Performance tests}
  \begin{itemize}
    \item Unit and integration tests usually only test the correctness of code
    \item Performance tests make sure the code fulfills requirements and does not get slower
    \item This introduction focuses on unit tests
    \item See the profiling lecture for more information on how to actually measure performance
  \end{itemize}
\end{frame}

\begin{frame}[c, fragile]{Example python code}
  We are going to use this simple function as example for our first unit tests:
  \inputcode[title=\texttt{examples/step1/fibonacci.py}]{./examples/step1/fibonacci.py}{python}
\end{frame}

\section{pytest}
\headlineframe{pytest}

\begin{frame}[c, fragile]{pytest}
  \begin{itemize}
    \item Standard framework for writing unit tests for Python projects
    \item Uses the \mintinline{python}+assert+ statement for tests
    \item Tests fail if an assertion fails or an exception is raised
    \item Uses introspection of the assertion to give detailed error messages
    \item Automatic test detection using patterns:
      \begin{itemize}
        \item Modules matching \mintinline{text}+test_*.py+ or \mintinline{text}+*_test.py+
        \item Functions called \mintinline{text}+test*+
        \item Methods named \mintinline{text}+test*+ of classes named \mintinline{text}+Test*+
      \end{itemize}

    \item Docs: \url{https://pytest.org}
  \end{itemize}

\end{frame}

\begin{frame}[c, fragile]{First Unit Test}
  \inputcode[title=\texttt{examples/step1/test\_fibonacci1.py}]{./examples/step1/test_fibonacci1.py}{python}

  \inputcode{./examples/pytest_output.txt}{text}
\end{frame}

\begin{frame}[c]{A note on imports}
  \inputcode[title=\texttt{examples/step1/test\_fibonacci1.py}]{./examples/step1/test_fibonacci1.py}{python}

  \begin{itemize}
    \item Against usual python style, you should import what you test in the test function
    \item Like this, the test discovery of pytest will also work when the import would fail and the failure is reported as part of the test
    \item Everything else, like standard library imports or third-party dependencies, is imported normally at the top
  \end{itemize}
\end{frame}

\begin{frame}[c]{Testing Exceptions}
  Make sure the correct exception is thrown, e.\,g.\ for invalid input:
  \inputcode[title=\texttt{examples/step2/fibonacci.py}]{./examples/step2/fibonacci.py}{python}[lastline=4]
  \inputcode[title=\texttt{examples/step2/test\_exception.py}]{./examples/step2/test_exception.py}{python}

  The same can be done for warnings using \mintinline{python}+pytest.warns+
\end{frame}


\begin{frame}[c, fragile]{Careful with floating point numbers}
  \inputcode[title={Naive, this fails}]{./examples/floating/test_floating.py}{python}[firstline=3, lastline=4]
  \inputcode[title={Correct approach, using \texttt{pytest.approx}}]{./examples/floating/test_floating.py}{python}[firstline=6]

  See \url{https://0.30000000000000004.com/}
\end{frame}

\begin{frame}[c, fragile]{Using numpy testing utitlities}
  \inputcode[title={Using numpy}]{./examples/floating/test_numpy.py}{python}
  See \url{https://numpy.org/doc/stable/reference/routines.testing.html}
\end{frame}

\begin{frame}[c, fragile]{Using astropy quantity support}
  \inputcode[title={Using astropy units}]{./examples/floating/test_astropy.py}{python}
\end{frame}

\begin{frame}[c]{Fixtures}
  \begin{itemize}
    \item Data and resources used by tests can be injected into tests using \enquote{fixtures}
    \item Fixtures are provided by functions decorated with \texttt{@fixture}
    \item Fixtures have a scope $⇒$ same object used per session, module, class or function
    \item Default is \mintinline{python}+scope="function"+
  \end{itemize}
  \inputcode{./examples/fixtures/test_fixtures.py}{python}
\end{frame}

\begin{frame}[c]{Fixtures provided by pytest}
  pytest provides several builtin fixtures for
  \begin{itemize}
    \item temporary directories \mintinline{python}+tmp_path+ / \mintinline{python}+tmp_path_factory+
    \item Testing output to stdout / stderr \mintinline{python}+capsys+
    \item Testing logging \mintinline{python}+caplog+
    \item Monkeypatching \mintinline{python}+monkeypatch+
  \end{itemize}
  More at \url{https://docs.pytest.org/en/6.2.x/fixture.html}
\end{frame}

\begin{frame}[c]{capsys – Fixture for testing the standard streams}
  \inputcode{./examples/fixtures/test_capsys.py}{python}
\end{frame}

\begin{frame}[c]{caplog – Fixture for testing logging}
  \inputcode{./examples/fixtures/test_logging.py}{python}
\end{frame}

\begin{frame}[c]{Temporary paths}
  \begin{itemize}
    \item For tests that need to create files, use the \mintinline{python}+tmp_path+ fixture \\
      $⇒$ Avoids cluttering and conflicts when running tests multiple times / between tests
    \item \mintinline{python}+tmp_path+ has scope \emph{function}, so each test gets its own temporary directory
    \item These directories are not cleaned up after the tests, so you can inspect the results
    \item If you need a temporary path with a wider scope, add a new fixture using \mintinline{python}+tmp_path_factory+
  \end{itemize}
\end{frame}

\begin{frame}[c]{Temporary paths}
  \inputcode{./examples/fixtures/test_tmp_paths.py}{python}

  Run the test and checkout \mintinline{bash}+/tmp/pytest-of-$USER/pytest-current/test_to_csvcurrent+
\end{frame}

\begin{frame}[c, fragile]{Fixtures that need a cleanup step}
  \begin{itemize}
    \item Sometimes, resources or data need to be cleaned up after the test have run
    \item This can be implemented using a generator fixture that yields the data and cleans up after the yield
  \end{itemize}

  \begin{code}{python}
    @pytest.fixture()
    def database_connection():
        connection = database.connect()
        yield connection
        # close after use
        connection.close()

    @pytest.fixture()
    def database_connection():
        # even better, with a context manager
        with database.connect() as connection:
            yield connection
  \end{code}
\end{frame}

\begin{frame}[c]{Parametrized Tests and Fixtures}
  \begin{itemize}
    \item Parametrization allows to run the same test on multiple inputs
    \item Very useful to reduce code repetition and get clearer messages
  \end{itemize}

  \inputcode[title={A parametrized test}]{./examples/step3/test_parametrized.py}{python}
\end{frame}


\begin{frame}[c, fragile]{Conditional tests}
  Some tests can only be run under specific conditions
    \begin{itemize}
      \item Tests for features requiring optional dependencies
        \inputcode[title={This test is skipped when numpy is not available}]{./examples/test_skipped.py}{python}[firstline=5, lastline=7]
      \item Tests for specific operating systems or versions
        \inputcode[title={This test is only executed on Windows}]{./examples/test_skipped.py}{python}[firstline=9]
    \end{itemize}
\end{frame}

\begin{frame}[c, fragile]{Expected failures}
  It sometimes makes sense to implement tests that are expected to fail:
  \begin{itemize}
    \item Planned but not yet implemented features
    \item Known but not yet fixed bugs
    \item These tests shouldn't make your whole test suite fail
  \end{itemize}

  \inputcode[title={This test is expected to fail}]{./examples/test_xfail.py}{python}
\end{frame}

\begin{frame}[c, fragile]{Choosing which tests to run}
  pytest offers fine-grained control over which tests to run

  \begin{itemize}
    \item Select a specific test:
      \begin{code}{bash}
        $ pytest test_module.py::test_name
      \end{code}
    \item Run only tests that failed the last time pytest was run
      \begin{code}{bash}
        $ pytest --last-failed
      \end{code}
  \end{itemize}

  \begin{columns}[onlytextwidth, c]%
    \begin{column}{0.475\textwidth}%
      \begin{itemize}
        \item Stop after N failures
      \end{itemize}
    \end{column}%
    \hfill%
    \begin{column}{0.475\textwidth}%
      \begin{code}{bash}
        $ pytest --maxfail=2
      \end{code}
    \end{column}%
  \end{columns}%

  \begin{columns}[onlytextwidth, c]%
    \begin{column}{0.475\textwidth}%
      \begin{itemize}
        \item Using matching expressions
      \end{itemize}
    \end{column}%
    \hfill%
    \begin{column}{0.475\textwidth}%
      \begin{code}[nobeforeafter, width=\linewidth, box align=base]{bash}
        $ pytest -k "fib"
      \end{code}
    \end{column}%
  \end{columns}%

  \begin{columns}[onlytextwidth, c]%
    \begin{column}{0.475\textwidth}%
      \begin{itemize}
        \item Run tests for an installed package
      \end{itemize}
    \end{column}%
    \hfill%
    \begin{column}{0.475\textwidth}%
      \begin{code}[nobeforeafter, width=\linewidth, box align=base]{bash}
        $ pytest --pyargs fibonacci
      \end{code}
    \end{column}%
  \end{columns}%
\end{frame}

\begin{frame}[c, fragile]{Choosing which tests to run – Using markers}
  \begin{itemize}
    \item Define markers in \texttt{pyproject.toml}
      \begin{code}{toml}
        [tool.pytest.ini_options]
        markers = ["slow"]
      \end{code}
    \item Add the marker to a test
      \inputcode{./examples/test_marks.py}{python}[firstline=4, lastline=7]
    \item Run tests using marker expressions
      \begin{code}{bash}
        $ pytest -m "not slow"
        $ pytest -m "slow"
      \end{code}
  \end{itemize}
\end{frame}

\begin{frame}[c, fragile]{Debugging}
  \begin{itemize}
    \item Unit tests can be very useful for debugging

    \item E.\,g.\ Write a new test that triggers the bug → investigate → make it pass

    \item pytest allows you to jump into pdb when a test fails:
      \begin{code}{bash}
        $ pytest --pdb
      \end{code}

    \item or any other debugger, e.\,g.\ ipython's:
      \begin{code}{bash}
        $ pytest --pdb --pdbcls=IPython.terminal.debugger:TerminalPdb
      \end{code}
  \end{itemize}
\end{frame}

\section{Test Coverage}
\headlineframe{Test Coverage}

\begin{frame}[c, fragile]{Test Coverage}
  \begin{itemize}
    \item Test coverage is a metric measuring how much of the code is tested:
      \begin{equation*}
        \operatorname{coverage} = \frac{\text{Lines of code executed during tests}}{\text{Total lines of code}}
      \end{equation*}
    \item Can be helpful to find parts of code that are not tested (enough).
    \item Especially useful in CI system to check that new / changed code is tested
    \item One more badge \faSmileWink[regular]{}! \includegraphics[height=0.5cm]{badge.pdf}
  \end{itemize}
\end{frame}

\begin{frame}[c, fragile]{Create a coverage report}
  \begin{itemize}
    \item Print coverage after test suite
      \begin{code}{bash}
        $ pytest --cov=fibonacci
      \end{code}
    \item Create a detailed report in html format
      \begin{code}{bash}
        $ pytest --cov=fibonacci --cov-report=html
      \end{code}
    \item Serve the report using python's built-in http server and explore in the browser:
      \begin{code}{bash}
        $ python -m http.server -d htmlcov
      \end{code}
  \end{itemize}
\end{frame}

\begin{frame}[c, fragile]{Limitations of line coverage}
  Executed number of lines of code are not a perfect measure.
  \begin{code}{python}
    if some_condition is True:
        do_stuff()
    do_other_stuff()
  \end{code}

  \onslide<2->{%
    When during the tests \mintinline{python}+some_condition is True+, this code will have 100\,\% coverage.
    \bigskip
    But what about \mintinline{python}+some_condition is not True+?
  }

  \onslide<3->

  \begin{code}{python}
    result = scipy.optimize.minimize(likelihood, ...)
  \end{code}

  \onslide<4->
  Calling functions from other packages can have arbitrarily many branches
  \begin{code}[title={Run pytest with branch coverage}]{bash}
    $ pytest --cov=fibonacci --cov-report=html --cov-branch
  \end{code}
\end{frame}

\section{Mocking / Monkeypatching}
\headlineframe{Mocking / Monkeypatching}
\begin{frame}[c]{Mocking / Monkeypatching}
  \begin{itemize}
    \item Sometimes, classes or functions have behaviour that prevents unit testing
    \item E.\,g.\ code that speaks to specific hardware, makes web requests, relies on system time ...
    \item This is usually a sign of insufficient modularization / separation of concerns
    \item A solution can be mocking or monkeypatching, if it is not possible to improve the actual code
  \end{itemize}
\end{frame}

\begin{frame}[c]{Mocking / Monkeypatching}
  \inputcode{./examples/test_monkeypatching.py}{python}
\end{frame}

\section{Test Driven Development}
\headlineframe{Test Driven Development}

\begin{frame}[c]{Test Driven Development (TDD)}
  \begin{itemize}
    \item Test Driven Development is a powerful paradigm
    \item Essentially, to implement a new feature
      \begin{enumerate}
        \item Write the tests before any implementation code
        \item Run the tests → they should all fail
        \item Write the minimal implementation that makes the test pass
        \item All tests should now pass
        \item Cleanup, refactor, tests must keep passing
      \end{enumerate}
  \end{itemize}
\end{frame}

\begin{frame}[c]{Test Drive Development}
  \begin{itemize}
    \item TDD forces you to think about requirements and API \emph{before} writing the actual code
    \item Especially usefull when
      \begin{itemize}
        \item you have clear specifications
        \item investigating / trying to fix a bug
        \item working on a new greenfield project
      \end{itemize}
    \item Not so easy to use when
      \begin{itemize}
        \item working in a large, historic codebase without good test coverage
        \item doing explorative work
      \end{itemize}
  \end{itemize}
\end{frame}

\section{Doctests}
\headlineframe{Doctests}

\begin{frame}[c, fragile]{Doctests}
  \begin{itemize}
    \item Examples are an important part of every documentations
      \inputcode{../fibonacci/fibonacci/__init__.py}{python}[firstline=1, lastline=8]
    \item Important to verify that the examples stay up to date and are correct
    \item Solution: run all the examples and check the expected output
      \begin{code}{bash}
        $ pytest --doctest-glob="*.rst" --doctest-modules
      \end{code}
    \item This will find and execute code blocks in docstrings and documentation rst-files
    \item Checks the output is what is expected
  \end{itemize}
\end{frame}

\section{Continuous Integration}
\headlineframe{Continuous Integration}
\begin{frame}[c]{Continuous Integration}
  \begin{itemize}
    \item CI systems run the build, unit tests and code quality checks automatically
    \item They should run for each push event / opened pull request
    \item All architectures, operating systems and versions you support should be tested
    \item Many tools provide detailed reporting that help with code reviews
    \item You should require the passing of the CI system for pull requests
    \item All providers also support encrypted secrets for confidential information\\
      \begin{itemize}
        \item Automatize upload of releases
        \item Access private data needed for tests
        \item ...
      \end{itemize}
  \end{itemize}
\end{frame}

\begin{frame}[c]{Providers}
  \begin{description}[Github Actions]
    \item[GitHub Actions] \url{https://docs.github.com/en/actions}
      \begin{itemize}
        \item Free for all public GitHub repositories
        \item Linux, Mac and Windows builds
        \item Support for custom runners that you can self-host
        \item Recommended for projects on GitHub
      \end{itemize}
    \item[GitLab CI] \url{https://docs.gitlab.com/ee/ci/}
      \begin{itemize}
        \item 400 minutes of build time per month with gitLab.com Free
        \item Support for custom runners that you can self-host
        \item Also available for self-hosted GitLabs (you have to setup at least one runner)
      \end{itemize}
    \item[Jenkins] \url{https://www.jenkins.io/}
      \begin{itemize}
        \item Open-Source CI platform you can self-host
      \end{itemize}
  \end{description}

  Many more, including \href{https://travis-ci.com/}{Travis CI}, \href{https://www.appveyor.com/}{AppVeyor}, \href{https://circleci.com/}{circle\textbf{ci}}, \href{https://azure.microsoft.com/de-de/services/devops/pipelines/}{Azure Pipelines}
\end{frame}

\begin{frame}[c]{Defining a CI Workflow}
  \begin{itemize}
    \item All of these providers use different configuration files to specify a workload
    \item Despite these differences, the idea is always the same
      \begin{itemize}
        \item Define environments, operating systems, software versions
        \item Get the code
        \item Install dependencies
        \item Compile / build / install the software
        \item Run the tests
        \item Upload results
      \end{itemize}
  \end{itemize}
\end{frame}

\begin{frame}[c]{A minimal github actions example}
  \begin{itemize}
    \item See the demo project at \url{https://github.com/maxnoe/pyfibonacci}
  \end{itemize}
\end{frame}


\begin{frame}[c]{Some CI helpful Tools}
  \begin{description}[\url{https://reviewnb.com/}]
    \item[\url{https://codecov.io/}] Explore coverage reports and adds coverage checks to pull requests
    \item[\url{https://reviewnb.com/}] Makes reviewing jupyter notebooks easier / possible
    \item[\url{https://codacy.com/}] Static code analysis, e.\,g.\ style linting
  \end{description}
\end{frame}
\end{document}
